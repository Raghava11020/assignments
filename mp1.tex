\documentclass[12pt]{article}
\usepackage{amsmath,amsfonts,amssymb,float}
\title{NCERT DISCRETE}
\author{Raghava Ganji\\EE23BTECH11020}
\date{}
\parindent 0px
\begin{document}
\maketitle
\textbf{Question 10.5.3.5:}
The first term of an AP is $5$, the last term is $45$ and the sum is $400$. Find the number of terms and the common difference.\\
\textbf{solution:}\\
Given AP is 5, \ldots, 45.\\
Therefore $x_1=5$ and $x_n=45$ and given $S_n$=400.\\
Let us assume n is the number of terms and d is the common difference of the given AP.\\\\
\begin{table}[H]
\centering
\begin{tabular}{|c|c|c|}\hline
$x_1$ & $x_n$ & $S_n$\\ \hline
$5$ & $45$ & $400$\\ \hline
\end{tabular}
\caption{Given inputs}
\end{table}
\begin{table}[H]
\centering
\begin{tabular}{|c|c|}\hline
n & d\\ \hline
$?$ &$?$\\ \hline
\end{tabular}
\caption{Finding Variables}
\end{table}
we know the equation of $x_n$ is
\begin{equation}
\tag{1}
x_n=x_1+(n-1)d
\end{equation}
by substituting the values of $x_1$ and $x_n$ in the equation 1, we get
\begin{equation}
\tag{2}
40=(n-1)d
\end{equation}
Also we know the equation of $S_n$ is
\begin{equation}
\tag{3}
S_n=\frac{n}{2} [2x_1+(n-1)d]
\end{equation}
by substituting the values of $S_n$, $x_1$ and (n-1)d in the equation 3, we get
\begin{equation}
\tag{4}
400=\frac{n}{2} [10+40]
\end{equation}
\begin{equation}
\tag{5}
800=50n
\end{equation}
Therefore,
\begin{equation}
\tag{6}
n=16
\end{equation}
by substituting the value of n in the equation 2, we get
\begin{equation}
\tag{7}
40=15d
\end{equation}
\begin{equation}
\tag{8}
d=\frac{8}{3}
\end{equation}
Therefore the number of terms n=$16$ and the common difference d=$\frac{8}{3}$.
\end{document}
